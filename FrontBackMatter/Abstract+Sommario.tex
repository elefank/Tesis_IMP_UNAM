%*******************************************************
% Abstract+Sommario
%*******************************************************

\pdfbookmark{Abstract}{Abstract}
\begingroup
\let\clearpage\relax
\let\cleardoublepage\relax
\let\cleardoublepage\relax

%\chapter*{Abstract}
%\selectlanguage{american}
	%The volumetric and transport characterization of reservoir fluids is crucial within all sorts of industry processes, such as production, conditioning, refining, etc. Currently on Mexico there is a pressing need for reliable volumetric and transport properties of crude oils, particularly for heavy oils. Considering that heavy oils are of strategic importance for Mexico, it is equally important the development of accurate and reliable methodologies for the density and viscosity characterization of these fluids for the full range of conditions that they undergo.
	
	%Being the viscosity a key property for the oil industry, its measurement requires as a fundamental input parameter an accurate density measurement within the same range viscosity is measured.
	
	%The main purpose of this work is the design, validation and implementation of an experimental system operating in a range of temperature that goes from ambient to reservoir conditions $120$ \celsius and up to $1000$ bar of pressure to measure density of hydrocarbons within reference accuracy. For this purpose, the design of an hydraulic system for high pressure and high temperature was proposed and implemented; this system was developed to allow simultaneous measurements of viscosity and density of a sample.
	
	%The density measurement system includes different laboratory equipment; a high pressure syringe pump, transfer fluid piston cylinder, a high pressure vibrating U-tube Anton Paar densimeter capable of delivering better than four figures of precision, a thermal bath, all also connected to a high pressure falling body viscometer.
	
	%The experimental setting is calibrated and validated against five reference fluids.
	
	%Once experimental system has been validated with the reference fluids, three representative Mexican crude oils are measured: a heavy oil, a medium oil , and a light oil, all form the Gulf of Mexico. Our density data are compared with another results from other laboratory. A thorough uncertainly analysis is carried in order to estimate, as precisely as possible, the uncertainty related to our measurements. Our data is then used, in a related work, for the accurate estimation of the viscosity of the same fluids that are first analysis in this work.

\vfill

\selectlanguage{spanish}
\pdfbookmark[1]{Sumario}{Sumario}
\chapter*{Sumario}
%Il pacchetto modifica alcuni aspetti tipografici dello stile \classicthesis, %di Andr� Miede. Permette di riprodurre la veste grafica della guida \emph{L'arte di scrivere con \LaTeX}~\citep{pantieri:art}. Lo spunto per l'originale rielaborazione di \classicthesis{} mi  stato offerto da Daniel Gottschlag. Il pacchetto \`e stato scritto per il Gruppo Utilizzatori Italiani di \TeX{} e \LaTeX{} (\url{http://www.guitex.org/}).

La caracterización volumétrica y el transporte de los fluidos del yacimiento es crucial en todo tipo de procesos en la industria como la producción, acondicionamiento, refinación, etc. Actualmente en México hay una necesidad apremiante de conocer con precisión los datos de propiedades volumétricas y de transporte de los crudos, en particular para los aceites pesados. Teniendo en cuenta qué los aceites pesados, son de importancia estratégica para México, es igualmente importante el desarrollo de metodologías precisas y confiables para la caracterización de la densidad y viscosidad a lo largo de toda la gama de condiciones que experimentan durante su proceso de producción.

Siendo la viscosidad una propiedad clave para la industria del petróleo, su medición requiere como parámetro fundamental de entrada una medición de la densidad exacta dentro del mismo rango de presión y temperatura.

El objetivo principal de este trabajo es el diseño, validación e implementación de un sistema experimental que funciona en un rango que va desde la temperatura ambiente hasta condiciones de yacimiento $ 120 $ \celsius, y hasta $ 1000 $ bar de presión para medir la densidad de los hidrocarburos con precisión de referencia. Para este propósito, se propuso y se implementó el diseño de un sistema hidráulico de alta presión y alta temperatura; este sistema fue desarrollado para permitir mediciones simultáneas de viscosidad y densidad de una muestra.

El sistema de medición de densidad incluye diferentes equipos de laboratorio; una bomba de jeringa de alta presión, un cilindro de transferencia de pistón flotante, un densímetro \emph{Anton Paar} de tubo vibrante, capaz de entregar mas de cuatro figuras de precisión, un baño termostato recirculador, también conectado a un viscosímetro de cuerpo deslizante de alta presión.

El arreglo experimental está calibrado y validado en contra de cinco fluidos de referencia. Los fluidos de referencia elegidos son agua (destilada y des ionizada), tolueno con pureza de $99.9$\%, y tres mezclas de polímeros con valores certificados de densidad y viscosidad por el instituto de estándares de Alemania. Estos fluidos fueron seleccionados cuidadosamente para todo el rango de condiciones por presentar viscosidades similares a las observadas en muestras de aceite crudo extrapesado, pesado y medio.

Una vez que el sistema experimental se ha validado con los fluidos de referencia, se miden tres aceites crudos representativos mexicanos: un aceite extra pesado, un aceite medio, y un aceite ligero todos provenientes del Golfo de México. Un análisis de incertidumbre se realiza con el fin de estimar, la incertidumbre relacionada con nuestras mediciones. Luego nuestros datos de densidad se comparan con los resultados de otra metodología. Nuestros datos se utiliza a continuación, en un trabajo relacionado, para la estimación precisa de la viscosidad de los mismos fluidos.

\endgroup			


\vfill